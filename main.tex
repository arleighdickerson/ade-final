\documentclass[a4paper]{article}

\usepackage[english]{babel}
\usepackage[utf8]{inputenc}
\usepackage{amsmath}
\usepackage{amssymb}
\usepackage{graphicx}

\title{Floquet Systems}

\author{Arleigh Dickerson}

\date{\today}

\begin{document}
\maketitle

\section{Introduction}

Bla bla bla bla.

\section{Theorem 2.72}

\subsection{Theorem}

Let $\mu_1$, $\mu_2$,\ldots, $\mu_n$ be the Floquet multipliers of the Floquet system $x' = A(t)x$. Then the trivial solution is \begin{enumerate}
    \item globally asymptotically stable on $[0,\infty)$
    \item stable on $[0,\infty)$ provided $|\mu_i| \leq 1$, $1 \leq i \leq \infty$ and whenever $|\mu_i| = 1$, $\mu_i$ is a simple eigenvalue
    \item unstable on $[0,\infty)$ provided there is an $i_0$, $1 \leq i_0 \leq n$ such that $|\mu_{i_0}| > 1$.
\end{enumerate}

\subsection{Proof}
The following proof is for the two-dimensional case. In the proof of Floquet's theorem, $B$ was picked such that $e^{B \omega} = C$. 
Also, notice that the Jordan canonical form theorem gives us matrices $M$ and $J$ such that $B = MJM^{-1}$, where either $J = \begin{bmatrix}
    \rho_1 & 0 \\
    0 & \rho_2
\end{bmatrix}$ or $J = \begin{bmatrix}
    \rho_1 & 1 \\
    0 & \rho_1
\end{bmatrix}$ where $\rho_1$, $\rho_2$ are the eigenvalues of $B$. It follows that (show why, properties of $e^A$)
\begin{equation*}
\begin{split}
    C &= e^{B\omega}\\
      &= e^{MJM^{-1}}\\
      &= M e^{J\omega} M^{-1}\\
      &= M K M^{-1}\\
\end{split}
\end{equation*}
where either $K = \begin{bmatrix}
    e^{\rho_{1}\omega} & 0 \\
    0 & e^{\rho_{2}\omega}\\
\end{bmatrix}$, or $K = \begin{bmatrix}
    e^{\rho_{1}\omega} & \omega e^{\rho_{1}\omega}\\
    0 & e^{\rho_{1}\omega}\\
\end{bmatrix}$.
Since the eigenvalues of $K$ are the same (show why exercise 2.14) as the eigenvalues of $C$, we get that the Floquet multipliers are $\mu_i = e^{\rho_{i}\omega}$ with $i = 1,2$ where it is possible that $\rho_1 = \rho_2$. Since $|\mu_i| = e^{Re(\rho_i)\omega}$, we have that
\begin{equation*}
\begin{split}
    |\mu_i| < 1 &\text{ iff } Re(\rho_i) < 0 \\
    |\mu_i| = 1 &\text{ iff } Re(\rho_i) = 0 \\
    |\mu_i| > 1 &\text{ iff } Re(\rho_i) > 0\text{.} \\
\end{split}
\end{equation*}

By theorem 2.71 the equation 
\begin{equation*}
    x(t) = P(t)y(t)
\end{equation*}
gives a one-to-one correspondence between solutions of the Floquet system $x' = A(t)x$ and $y' = By$. Note that there is a constant $Q_1 > 0$ so that (what kind of norm is this?) 
\begin{equation*}
    \Vert x(t) \Vert = \Vert P(t)y(t) \Vert \leq \Vert P(t) \Vert \Vert y(t) \Vert  \leq Q_{1} \Vert y(t) \Vert \text{,}
\end{equation*}
for $t \in \mathbb{R}$ and since $y(t) = P^{-1}(t)x(t)$ there is a constant $Q_2 > 0$ such that 
\begin{equation*}
    \Vert y(t) \Vert = \Vert P^{-1}(t)x(t) \Vert \leq \Vert P^{-1}(t) \Vert \Vert x(t) \Vert  \leq Q_{2} \Vert x(t) \Vert \text{,}
\end{equation*}
for $t \in \mathbb{R}$.

\section{Theorem 2.73}

\subsection{Theorem}

The number $\mu_0$ is a Floquet multiplier of the Floquet system $x' = A(t)x$ if and only if there is a nontrivial solution $x$ such that $x(t + \omega) = \mu_0 x(t)$ for all $t \in \mathbb{R}$. Consequently, the Floquet system has a nontrivial periodic solution of period $\omega$ if and only if $\mu_0 = 1$ is a floquet multiplier.

\subsection{Proof}

\end{document}


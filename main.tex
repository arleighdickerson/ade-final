\documentclass[a4paper]{article}

\usepackage[english]{babel}
\usepackage[utf8]{inputenc}
\usepackage{amsmath}
\usepackage{amssymb}
\usepackage{graphicx}

\title{Advanced Differentials Final Paper}

\author{Arleigh Dickerson}

\date{\today}

\begin{document}
\maketitle

\section{Introduction}
Hello, friends. Today we will be learning about trivial and nontrivial solutions to Floquet systems. Let's begin.
\section{Theorem 2.72}

In this theorem, we see some of the qualitative properties of the trivial solution. Specifially, this theorem examines how the value of Floquet multipliers influences the trivial solution's stability or lack thereof. To accomplish this end, we will eventually break our argument up into three cases based upon the value of $|\mu_i|$ and link it to the Stability Theorem (2.49).

\subsection{Theorem}

Let $\mu_1$, $\mu_2$,\ldots, $\mu_n$ be the Floquet multipliers of the Floquet system $x' = A(t)x$. Remember from previously that for a $\mu$ to be a Floquet multiplier of a system, it must be an eigenvalue of $C := \Phi^{-1}(0)\Phi(\omega)$. Our claim is that the trivial solution is \begin{enumerate}
    \item globally asymptotically stable on $[0,\infty)$
    \item stable on $[0,\infty)$ provided $|\mu_i| \leq 1$, $1 \leq i \leq \infty$ and whenever $|\mu_i| = 1$, $\mu_i$ is a simple eigenvalue
    \item unstable on $[0,\infty)$ provided there is an $i_0$, $1 \leq i_0 \leq n$ such that $|\mu_{i_0}| > 1$.
\end{enumerate}

\subsection{Proof}
The following proof is for the two-dimensional case. In proving Floquet's theorem, we chose an arbitrary $B$ such that $e^{B \omega} = C$. 
Also, notice that the Jordan canonical form theorem gives us matrices $M$ and $J$ such that $B = MJM^{-1}$, where either $J = \begin{bmatrix}
    \rho_1 & 0 \\
    0 & \rho_2
\end{bmatrix}$ or $J = \begin{bmatrix}
    \rho_1 & 1 \\
    0 & \rho_1
\end{bmatrix}$ where $\rho_1$, $\rho_2$ are the eigenvalues of $B$. Recall that if a matrix $Y$ is invertible, then $e^{Y X Y^{-1}} = Y e^{X} Y^{-1}.$
With this property of the matrix exponential function in mind, it follows that
\begin{equation*}
\begin{split}
    C &= e^{B\omega}\\
      &= e^{MJM^{-1}}\\
      &= M e^{J\omega} M^{-1}\\
      &= M K M^{-1}\\
\end{split}
\end{equation*}
where either $K = \begin{bmatrix}
    e^{\rho_{1}\omega} & 0 \\
    0 & e^{\rho_{2}\omega}\\
\end{bmatrix}$, or $K = \begin{bmatrix}
    e^{\rho_{1}\omega} & \omega e^{\rho_{1}\omega}\\
    0 & e^{\rho_{1}\omega}\\
\end{bmatrix}$.
Since the eigenvalues of $K$ are the same (show why exercise 2.14) as the eigenvalues of $C$, we get that the Floquet multipliers are $\mu_i = e^{\rho_{i}\omega}$ with $i = 1,2$ where it is possible that $\rho_1 = \rho_2$. Since $|\mu_i| = e^{Re(\rho_i)\omega}$, we have that
\begin{equation*}
\begin{split}
    |\mu_i| < 1 &\text{ iff } Re(\rho_i) < 0 \\
    |\mu_i| = 1 &\text{ iff } Re(\rho_i) = 0 \\
    |\mu_i| > 1 &\text{ iff } Re(\rho_i) > 0\text{,} \\
\end{split}
\end{equation*} where the $Re$ function discards the complex portion of a value and returns only the real part.

As explained by Kelly, the equation from Theorem 2.71 
\begin{equation*}
    x(t) = P(t)y(t)
\end{equation*}
establishes a bijection between solutions of the Floquet system $x' = A(t)x$ and $y' = By$. Note that there is a constant $Q_1 > 0$ so that 
\begin{equation*}
    \Vert x(t) \Vert = \Vert P(t)y(t) \Vert \leq \Vert P(t) \Vert \Vert y(t) \Vert  \leq Q_{1} \Vert y(t) \Vert \text{,}
\end{equation*}
for $t \in \mathbb{R}$ and since $y(t) = P^{-1}(t)x(t)$ there is a constant $Q_2 > 0$ such that 
\begin{equation*}
    \Vert y(t) \Vert = \Vert P^{-1}(t)x(t) \Vert \leq \Vert P^{-1}(t) \Vert \Vert x(t) \Vert  \leq Q_{2} \Vert x(t) \Vert \text{,}
\end{equation*}
for $t \in \mathbb{R}$. To conclude this proof, we must use some points from the Stability Theorem (2.49). The theorem states that \begin{enumerate}
    \item If $A$ has an eigenvalue with positive real part, then the trivial solution is unstable on $[0,\infty)$.
    \item If all the eigenvalues of $A$ with zero real parts are simple (multiplicity one) and all other eigenvalues of A have negative real parts, then the trivial solution is stable on $[0,\infty)$.
    \item If all the eigenvalues of $A$ have negative real parts, then the trivial solution of $x' = Ax$ is globally asymptotically stable on $[0,\infty)$.
\end{enumerate}

\section{Theorem 2.73}

DESCRIBE WHAT WE'RE GOING TO DO AND HOW IT IS DONE

\subsection{Theorem}

The number $\mu_0$ is a Floquet multiplier of the Floquet system $x' = A(t)x$ if and only if there is a nontrivial solution $x$ such that $x(t + \omega) = \mu_0 x(t)$ for all $t \in \mathbb{R}$. Consequently, the Floquet system has a nontrivial periodic solution of period $\omega$ if and only if $\mu_0 = 1$ is a floquet multiplier.

\subsection{Proof}

By assuming $\mu_0$ is a Floquet multiplier of $x' = A(t)x$, we can see that (SHOW THIS) $\mu_0$ is an eigenvalue of 
\begin{equation*}
    C := \Phi^{-1}(0)\Phi(\omega)
\end{equation*}
where $\Phi$ is a fundamental matrix of $x' = A(t)x$. Let $x_0$ be an eigenvector corresponding to $\mu_0$ and define the vector function $x$ by $x(t) = \Phi(t)x_0$ with $t \in \mathbb{R}$.

Then $x$ is a nontrivial solution of $x' = A(t)x$ and \begin{equation*}
\begin{split}
    x(t + \omega) &= \Phi(t + \omega) x_0 \\
                  &= \Phi(t) C x_0\\
                  &= \Phi(t) \mu_0 x_0 \\
                  &= \mu_0 x(t) \\
\end{split}
\end{equation*}
for all $t \in \mathbb{R}$.

Conversely, let us assume that there is a nontrivial solution $x$ such that \begin{equation*}
    x(t + \omega) = \mu_0 x(t)
\end{equation*}
for all $t \in \mathbb{R}$. Let $\Psi$ be a fundamental matrix of our Floquet system, then \begin{equation*}
    x(t) = \Psi(t) y_0
\end{equation*}
for all $t \in \mathbb{R}$ and some nontrivial vector $y_0$. By Floquet's theorem the matrix function $\Psi(\cdot + \omega)$ is also a fundamental matrix. Hence $x(t + \omega) = \mu_0 x(t)$, so $\Psi(t + \omega)y_0 = \mu_0 \Psi(t) y_0$. Therefore,\begin{equation*}
    \Psi(t) D y_0 = \Psi(t) \mu_0 y_0
\end{equation*}
where $D := \Psi^{-1}(0) \Psi(\omega)$. It follows that $D y_0 = \mu_0 y_0$, and so $\mu_0$ is an eigenvalue of $D$ and therefore a Floquet multiplier of our Floquet system.
\end{document}

